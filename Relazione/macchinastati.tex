\section{Macchina a stati finiti}
Come riferimento in figura \ref{msf} sono stati individuati 4 stati per la gestione dell'automa. Ogni stato chiama le funzioni che vengono eseguite per espletare il proprio compito ed appositi metodi regolano la transizione da uno stato all'altro al verificarsi di specifiche condizioni.\\ 
\emph{Line Follower} e \emph{Manual Control} rappresentano i due stati principali che permettono rispettivamente una guida autonoma sul tracciato o un controllo manuale.
Lo stato \emph{Idle} è uno stato di attesa del robot in cui in ogni momento è possibile accedervi su decisione dell'utente e in cui lo stesso può scegliere quale modalità di funzionamento adottare.
Lo stato \emph{Init}, infine, è solamente uno stato di inizializzazione all'avvio del robot. 
\begin{figure}[h!t]
\centering
\subfloat[][\emph{schema concettuale}]
   {\begin{tikzpicture}[->, >=stealth', shorten >=1pt, auto, node distance=2.5cm, semithick]

%====Definizione stile stati================================================

	\tikzstyle{every state}=[fill, draw=none, orange ,text=white]
  	\tikzstyle{accepting}=[green!50!black, text=white]
  	\tikzstyle{manual}=[blue!50!black, text=white]
  	\tikzstyle{initial}= [red, text=white]

%====Definizione stati e posizione===========================================
	\node[state,initial] 			(A)                    		{\textsc{A}};
 	\node[state]          			(B) [right of=A]		{\textsc{B}};
 	\node[]							(B1)[right of=B]		{};
 	\node[state,accepting]	(C) [above of=B1, node distance=2.5 cm]		{\textsc{C}};
  	\node[state,manual]       (D) [below of=B1, node distance=2.5 cm]  	{\textsc{D}};
  	
%====Definizione collegamenti e stile frecce================================================================
      			 %from     stile                      scritte       	to		
      \path 	(A)		edge	[right]									(B)
      				(B)		edge	[bend left]							(C)
      				(C)		edge	[bend left]							(B)
      				(B)		edge[bend left]							(D)
      				(D)		edge[bend left]							(B);    
\end{tikzpicture}} \qquad\quad
\subfloat[][\emph{legenda}]{\begin{tikzpicture}[->, >=stealth', shorten >=1pt, auto, node distance=1.5cm, semithick]

%====Definizione stile stati================================================

	\tikzstyle{every state}=[fill, draw=none, orange ,text=white]
  	\tikzstyle{accepting}=[green!50!black, text=white]
  	\tikzstyle{manual}=[blue!50!black, text=white]
  	\tikzstyle{initial}= [red, text=white]

%====Definizione stati e posizione===========================================

	\node[state,initial] 		(A) [label = 0:Init ]                  		{\textsc{A}}; 	
 	\node[state]          		(B) [below of= A, label = 0:Idle]		{\textsc{B}};		
 	\node[state,accepting](C) [below of= B, label = 0:Line Follower]	{\textsc{C}};
  	\node[state,manual]    (D) [below of= C, label = 0:Manual Control]{\textsc{D}};		

%====Definizione collegamenti e stile frecce================================================================
      			 %from     stile                      scritte       	to		
%      \path 	(A)		edge	[right]									(B)
%      				(B)		edge	[bend left]						(C)
%      				(C)		edge	[bend left]						(B)
%      				(C)		edge[right]								(D)
%      				(D)		edge[bend left]						(A);    
\end{tikzpicture}}
\caption{Macchina a stati finiti}
\label{msf}
\end{figure}

\subsection{Line Follower}
Nella modalità line follower in ogni istante viene effettuata la verifica della presenza ostacoli mediante sensore ad ultrasuoni.
Qualora non venisse riscontrato un pericolo di collisione, entro una prefissata distanza di sicurezza, l'algoritmo prosegue con la lettura del tracciato altrimenti impartisce al robot il comando di invertire il moto.
Al contempo la coppia di sensori di linea istantaneamente legge il tracciato e lo codifica secondo la tabella \ref{error_normali}, attribuendo un errore che serve a correggere la velocità dei motori mediante un controllo \textsc{PID}\footnote{Proporzionale-Integrativo-Derivativo, comunemente abbreviato come PID}.
\begin{table}[h]\footnotesize
\center
\begin{tabular}{llccr}
\label{normali}
                          			&			&DX  SX&	SX  DX    	& Errore\\
\hline \\
  GO FORWARD          		& ==  	&1, 2  	& O X  X O 	& 0\\% (line = nero & background = bianco) 
  TURN LEFT VERY SOFT 	&	==	&	3, 2	&	O X  O O  &	1\\
  TURN LEFT SOFT       	&	==  	&	1,	0	&	X X  X O   &	2\\
  TURN LEFT HARD     		&	==  	&	3, 0	&	X X  O O	&	3\\
  TURN LEFT VERYHARD  &	==  	&	3, 1	&	X O  O O	&	4\\
  TURN RIGHT VERY SOFT&	== 	& 1, 3	&	O O  X O  &	-1\\
  TURN RIGHT SOFT    		&	==  	&	0, 2	& 	O X  X X  	&	-2\\
  TURN RIGHT HARD   		&	== 	&	0, 3	&	O O  X X  	&	-3\\
  TURN RIGHT VERYHARD &	== 	&	2, 3	&	O O  O X	&	-4\\
  NO LINE           				&	== 	&	3, 3	&	O O  O O  &	 5\\
  CROSS              				&	== 	& 0, 0   &	X X  X X   	&  6\\
  \hline
\end{tabular}
\caption{Codifica input dei sensori di linea (X = tracciato nero; O = sfondo bianco)}
\label{error_normali}
\end{table}

\noindent Per garantire una maggiore robustezza al programma sono state considerate anche le casistiche eccezionali che coprono eventuali guasti o malfunzionamenti (ad esempio linea a contrasto invertito, sporco sui sensori o sul tracciato), come mostrato in tabella \ref{eccezioni}.

\begin{table}[h]\footnotesize
\center
\begin{tabular}{llccc}
\label{eccezioni}
								&			&	DX  SX	&	SX  	DX    &    Errore\\
 \hline\\
	GO FORWARD bis    	&	== 	&	2, 1		&	X O  O X	&  0\\
	EXCEPTION1       		&	==  	&	2, 0		&	X X  O X   &	1\\
  	EXCEPTION2        	&	== 	&	0, 1  	&	X O  X X   &  -1\\
 	EXCEPTION3     	 	&	==  	&	2, 2	 	&	O X  O X	&  0\\
 	 EXCEPTION4           &	==  &1 , 1      & X O  X O     &	 0\\         
 	\hline
\end{tabular}
\caption{Codifica \textsc{casi eccezionali} (X = tracciato nero; O = sfondo bianco)}
\label{eccezioni}
\end{table}

%\noindent Quando il robot raggiunge un incrocio si arresta, esce dallo stato di line follower e passa nello stato idle attendendo istruzioni dall'utente che può scegliere dalla maschera disponibile sul terminale, quindi inviare il comando tramite connessione bluetooth. Il robot torna nello stato di line follower.
%In condizioni totalmente manuale il robot risponde ai comandi inviati da bluetooth evitando gli ostacoli rilevati nella sola parte frontale.
%Il modo più efficiente di progettare un controllo di un robot di questo tipo è basato sul modello di macchina a stati finiti in cui ad ogni stato corrisponde una specifica funzione del robot.
%% inserire diagramma a stati e riferimento alla figura
%Lo stato è  caratterizzato da una funzione d'ingresso (\emph{enter function}), una funzione di aggiornamento (\emph{update function}), un funziona di uscita (\emph{exit function}).
%La transizione da uno stato all'altro è soggetta al soddisfacimento di una certa condizione.


%\subsection{Stato Init}
%Stato di inizializzazione della macchina all'avvio del robot. Garantisce lo stato di fermo dei motori mediante il richiamo della funzione di stop e poi si aggiorna con la funzione start che  invoca la transizione verso lo \emph{stato Idle}.
%
%
%\subsection{Stato Idle}
%In questo stato l'enter function richiama la funzione che stampa sul terminale i comandi disponibili per l'interazione con l'utente, poi passa nell'update function  che aspetta il comando dell'utente inviato tramite il bluetooth per la transizione nello stato successivo.

%\subsection{Stato Line Follower}
%Nel seguente stato la prima operazione che viene eseguita è una lettura mediante sensore ad ultrasuoni della presenza di eventuali ostocali sul tracciato.
%La distanza di sicurezza scelta sperimentalmente è di $15 \, cm$ valore che permette al robot di arrestarsi e ruotare su stesso evitando contatti con l'ostacolo.
%Verificata tale condizione inizia la lettura dei sensori di linea che decifrano il percorso secondo la logica presentata in tabella \ref{error_normali}. Casi eccezionali in tabella \ref{eccezioni}. Ad ogni casistica è stato associato un errore che attraverso un controllo PD appositamente 
%implementato fornisce il valore di correzione da inviare ai motori per rimanere sul tracciato.

%FUNZIONE %path_error
%  in questa funzione in ingresso vengono ricevuti
%  due interi ricavati dai metodi dei line follewer
%  e in uscita viene restituito un intero che individua la posizione del 
%  robot rispetto alla linea da seguire (error = output - setpoint).
%  Sulla base dei valori ricevuti dai sensori si ottengono i seguenti casi:\\
%\subsection{Stato Manual Command}