% Use only LaTeX2e, calling the article.cls class and 12-point type.

\documentclass[12pt,a4paper,english,italian,hidelinks]{article}
\usepackage{fancyhdr}
\usepackage{multirow}
\usepackage{multicol}
\usepackage[section]{placeins}

% === Impostazione dei font ===============================
\usepackage[T1]{fontenc}
\usepackage[utf8]{inputenc}
\usepackage[italian]{babel}
\usepackage{ae}
\usepackage{relsize}
\usepackage{csquotes} 
\usepackage{amsmath}
\usepackage{amsfonts}
\usepackage{mathdots}
\usepackage[colorlinks=true]{hyperref}
\hypersetup{
	bookmarksnumbered=true,
	linkcolor=black,
	citecolor=black,
	pagecolor=black,
	urlcolor=black,
}
\usepackage{verbatim}
\usepackage{alltt}

% Users of the {thebibliography} environment or BibTeX should use the
% scicite.sty package, downloadable from *Science* at
% www.sciencemag.org/about/authors/prep/TeX_help/ .
% This package should properly format in-text
% reference calls and reference-list numbers.

\usepackage{scicite}

% Use times if you have the font installed; otherwise, comment out the
% following line.

\usepackage{times}

% The preamble here sets up a lot of new/revised commands and
% environments.  It's annoying, but please do *not* try to strip these
% out into a separate .sty file (which could lead to the loss of some
% information when we convert the file to other formats).  Instead, keep
% them in the preamble of your main LaTeX source file.

% The following parameters seem to provide a reasonable page setup.

\topmargin 0.0cm
\oddsidemargin 0.2cm
\textwidth 16cm 
\textheight 21cm
\footskip 1.0cm

%The next command sets up an environment for the abstract to your paper.

\newenvironment{sciabstract}{%
\begin{quote} \bf}
{\end{quote}}

% If your reference list includes text notes as well as references,
% include the following line; otherwise, comment it out.

\renewcommand\refname{References and Notes}

% The following lines set up an environment for the last note in the
% reference list, which commonly includes acknowledgments of funding,
% help, etc.  It's intended for users of BibTeX or the {thebibliography}
% environment.  Users who are hand-coding their references at the end
% using a list environment such as {enumerate} can simply add another
% item at the end, and it will be numbered automatically.

\newcounter{lastnote}
\newenvironment{scilastnote}{%
\setcounter{lastnote}{\value{enumiv}}%
\addtocounter{lastnote}{+1}%
\begin{list}%
{\arabic{lastnote}.}
{\setlength{\leftmargin}{.22in}}
{\setlength{\labelsep}{.5em}}}
{\end{list}}

% Include your paper's title here

\title{Mobile Robot} 

% Place the author information here.  Please hand-code the contact
% information and notecalls; do *not* use \footnote commands.  Let the
% author contact information appear immediately below the author names
% as shown.  We would also prefer that you don't change the type-size
% settings shown here.

\author
{Francesco Argentieri	\footnote{ID: 183892 mail: francesco.argentieri@studenti.unitn.it}, 
 Alessandro Luchetti	\footnote{ID: 111111 mail: @studenti.unitn.it},
 Luca Nicolodi				\footnote{ID: 111111 mail: @studenti.unitn.it}\\
\\
\normalsize{Università Degli Studi di Trento, Dipartimento Ingegneria Industriale}\\
\normalsize{C.D.L.M Ingegneria Meccatronica}\\
\normalsize{Corso di Informatica e Programmazione}
}

% Include the date command, but leave its argument blank.

\date{}

% === Integrazione delle figure ===========================
\usepackage{graphicx}
\graphicspath{{./imgs/}}
\renewcommand{\figurename}{Fig.}
\usepackage{subfig}

% === Per gli algoritmi ===================================
\usepackage{algorithmicx}
\usepackage[ruled]{algorithm}
\usepackage{algpseudocode}

% === Per le tabelle ======================================
\usepackage{tabularx}

% === Per la bibliografia multicolonna ====================
\usepackage{etoolbox}
\patchcmd{\thebibliography}{\list}{\begin{multicols}{2}\smaller\list}{}{}
\appto{\endthebibliography}{\end{multicols}}
%==Per le figure latex======================================
%Package e librerie per TikZ e PGF, le librerie non sono tutte necessarie a questo documento LATEX.
\usepackage{tikz, fp, ifthen, fullpage}
     \usepackage{pgfmath}
     \usetikzlibrary{backgrounds}
     \usetikzlibrary{decorations.pathmorphing, backgrounds, fit, calc, through}
     \usetikzlibrary{arrows,chains,matrix,scopes}
     \usetikzlibrary{automata}
     \usetikzlibrary{shapes,decorations,shadows}
     \usetikzlibrary{fadings}
     \usetikzlibrary{patterns}
     \usetikzlibrary{mindmap}
     \usetikzlibrary{decorations.text}
     \usetikzlibrary{decorations.shapes}
      \usepackage{pgfplots}
\pgfplotsset{compat=1.13}

%%%%%%%%%%%%%%%%% END OF PREAMBLE %%%%%%%%%%%%%%%%

\begin{document} 
% Double-space the manuscript.

\baselineskip24pt

% Make the title.

\maketitle 

%===inizio capitoli=================================================

	\section*{Abstract}
Nel report si espongono i risultati raggiunti nello sviluppo del progetto \emph{Mobile Robot}, un robot mobile con azionamento differenziale a due ruote che presenta queste caratteristiche di base:
 LineFollower;
 Riconoscimento incroci nel tracciato;
Comandi da remoto.
Si è scelto di implementarne il funzionamento mediante \emph{macchina a stati finiti}.

%Details for the Course Project Report: The report should be organized with the following sections: 
%TITLE Names and Surnames of the group's components 
%ABSTRACT 
%DESCRIPTION OF THE PROBLEM 
%SOLUTION FOUND Describe the used algorithm/algorithms 
%CONCLUSIONS 
%BIBLIOGRAPHY 
%The maximum allowed number of pages is 6, every page that exceed this limit will be ignored. Pictures, chart and plots can be added to the report.

	\section*{Introduzione}
Nella prima sezione del report si presentano le scelte preliminari effettuate al fine di soddisfare al meglio le caratteristiche di base richieste. Nella seconda sezione si descrive la struttura del codice realizzato soffermandosi sui punti chiave che ne hanno determinato lo sviluppo. Infine nella conclusione si riassumono i risultati raggiunti.
\section{Scelte progettuali}
Il progetto può essere suddiviso in due parti: una, inerente alla scelta ed assemblaggio dei componenti meccanici ed elettronici, mentre l'altra relativa allo sviluppo del software di controllo.
Ciascuna di esse viene analizzata  successivamente nei prossimi paragrafi. 
\subsection{Hardware}
Non avendo vincoli specifici si è scelto di utilizzare un struttura semplice come riportata in figura \ref{struct}, ma che potesse al tempo stesso soddisfare tutte le funzionalità prefissate.%inserire riferimento figure.
\begin{figure}[h!t]
\centering
\subfloat[][\emph{vista frontale}]{
\includegraphics[width=.4\textwidth]{IMG_0759.jpg} }\quad
\subfloat[][\emph{scorcio}]{
\includegraphics[width=.4\textwidth]{IMG_0763.jpg}}
\caption{Struttura robot}
\label{struct}
\end{figure}
\\La parte elettronica è basata sulla scheda MeOrion baseboard, un modulo a ultrasuoni per il rilevamento degli ostacoli, due moduli per i sensori tipo MeLineFollower, un modulo bluetooth. In figura \ref{schema} è illustrato il dettaglio dei collegamenti.
La scelta di adottare una coppia di sensori di linea è stata fatta per ottenere una migliore lettura del percorso, per una maggiore capacità di identificazione di casi particolari, come ad esempio curve a 90$^{\circ}$ o gli incroci e per un miglior controllo dei motori.
\begin{figure}[h]
	\subfloat[][\emph{collegamenti}]{\includegraphics[width=.65\textwidth]{SCHEMA}}\quad
\subfloat[][\emph{lista dei compomenti}]{\begin{tikzpicture}[->, >=stealth', shorten >=1pt, auto, node distance=1cm, semithick]

%====Definizione stile stati================================================

	\tikzstyle{every state}=[fill, draw, white ,text=black]

%====Definizione stati e posizione===========================================

	\node[state](A) [label = 0:Me Orion ]                  			{1}; 	
 	\node[state](B) [below of= A, label = 0:Me Bluetooth]	{2};		
 	\node[state](C) [below of= B, label = 0:Line Follower]	{3};
  	\node[state](D) [below of= C, label = 0:Me Ultrasonic Sensor]{4};		
  	\node[state](E) [below of= D, label = 0:Batteria Li-Po]{5};	
  	\node[state](F) [below of= E, label = 0:Me Dual DC MotorDriver]{6};
  	\node[state](G) [below of= F, label = 0:Me DCMotor]{7};	
  	\end{tikzpicture}}
\caption{Componenti elettronici}
\label{schema}
\end{figure}		
\subsection{Software}
L'implementazione del software, basata sul C++, è il cuore del progetto. Ci si è avvalsi delle librerie proprietarie Makeblock, della classe macchina a stati finiti, precedentemente realizzata, e di metodi appositamente elaborati per una corretta gestione delle funzionalità richieste.
	\section{Soluzione del problema}
\label{solution}
Nel dettaglio si descrive il software del robot partendo dallo strato che interfaccia le periferiche con l'unità di elaborazione.
\subsection{Libreria Makeblock}
L'utilizzo di questa libreria è stato fondamentale perché rappresenta la base dei tre livelli e permette allo sviluppatore di istanziare i moduli fisici come oggetti e di renderli operativi fin da subito dato che viene richiesta solo la porta a cui sono collegati.
Utilizzando i metodi disponibili dalla classe è possibile ricavare gli input provenienti dai sensori, permettere la comunicazione con altri dispositivi mediante bluetooth, pilotare i motori.
\subsection{Libreria Macchina a stati finiti}
Questo secondo strato software è stato sviluppato durante le esercitazioni di laboratorio.

\begin{figure}[h!t]
\centering
\subfloat[][\emph{Schema concettuale}.]
   {\begin{tikzpicture}[->, >=stealth', shorten >=1pt, auto, node distance=2.5cm, semithick]

%====Definizione stile stati================================================

	\tikzstyle{every state}=[fill, draw=none, orange ,text=white]
  	\tikzstyle{accepting}=[green!50!black, text=white]
  	\tikzstyle{initial}= [red, text=white]

%====Definizione stati e posizione===========================================
	\node[state,initial] 			(A)                    		{\textsc{A}};
 	\node[state]          			(B) [right of=A]		{\textsc{B}};
 	\node[state,accepting]	(C) [right of=B]		{\textsc{C}};
  	\node[state]         			(D) [right of=C]  		{\textsc{D}};
  	
%====Definizione collegamenti e stile frecce================================================================
      			 %from     stile                      scritte       	to		
      \path 	(A)		edge	[right]									(B)
      				(B)		edge	[bend left]							(C)
      				(C)		edge	[bend left]							(B)
      				(C)		edge[right]									(D)
      				(D)		edge[bend left]							(A);    
\end{tikzpicture}} \qquad\quad
\subfloat[][\emph{Legenda}.]{\begin{tikzpicture}[->, >=stealth', shorten >=1pt, auto, node distance=1.5cm, semithick]

%====Definizione stile stati================================================

	\tikzstyle{every state}=[fill, draw=none, orange ,text=white]
  	\tikzstyle{accepting}=[green!50!black, text=white]
  	\tikzstyle{initial}= [red, text=white]

%====Definizione stati e posizione===========================================

	\node[state,initial] 			(A) [label = 0:Init ]                  		{\textsc{A}}; 	
 	\node[state]          			(B) [below of= A, label = 0:Start]		{\textsc{B}};		
 	\node[state,accepting]	(C) [below of= B, label = 0:Line Follower]		{\textsc{C}};
  	\node[state]         			(D) [below of= C, label = 0:Manual Control]  		{\textsc{D}};		

%====Definizione collegamenti e stile frecce================================================================
      			 %from     stile                      scritte       	to		
%      \path 	(A)		edge	[right]									(B)
%      				(B)		edge	[bend left]						(C)
%      				(C)		edge	[bend left]						(B)
%      				(C)		edge[right]								(D)
%      				(D)		edge[bend left]						(A);    
\end{tikzpicture}}
\caption{Macchina a stati finiti}
\label{msf}
\end{figure}

\subsection{Libreria}
Infine questa libreria gestisce la risposta del robot al variare degli input durante il suo funzionamento, rappresenta lo strato di interconnessione tra i due prima analizzati.
Questa classe inizializza un oggetto che ammette in input i dati provenienti del sensore a ultra suoni e dai sensori di linea: con il primo dato verifica la presenza di ostacoli nella parte frontale se assenti o distanti almeno $15 \, cm$ da il via libera a proseguire lungo a direzione altrimenti impartisce al robot il comando di ruotare sul proprio asse e proseguire nella direzione opposta.
Con il secondo input proveniente dai sensori di linea riconosce il percorso affrontato secondo una serie di casi codificati secondo la tabella \ref{error_normali} attribuendo un errore. Tale valore viene utilizzato per corregere la velocità dei motori tramite un controller \textsc{pd}, si descriverà in dettaglio nella sezione \ref{PID}.
Vengono inoltre codificati dei casi eccezionali consultabili nella tabella \ref{eccezioni} a cui è stato assegnato un errore $0$ perchè \dots

\begin{table}[h]
\center
\begin{tabular}{llccr}
\label{normali}
                             %      (X=nero; O=bianco)
                          			&			&DX  SX&	SX  DX    	& Errore\\
\hline \\
  GO FORWARD          		& ==  	&1, 2  	& O X  X O 	& 0\\				% (line = nero & background = bianco) 
  TURN LEFT VERY SOFT 	&	==	&	3, 2	&	O X  O O  &	1\\
  TURN LEFT SOFT       	&	==  	&	1,	0	&	X X  X O   &	2\\
  TURN LEFT HARD     		&	==  	&	3, 0	&	X X  O O	&	3\\
  TURN LEFT VERYHARD  &	==  	&	3, 1	&	X O  O O	&	4\\
  TURN RIGHT VERY SOFT&	== 	& 1, 3	&	O O  X O  &	-1\\
  TURN RIGHT SOFT    		&	==  	&	0, 2	& 	O X  X X  	&	-2\\
  TURN RIGHT HARD   		&	== 	&	0, 3	&	O O  X X  	&	-3\\
  TURN RIGHT VERYHARD &	== 	&	2, 3	&	O O  O X	&	-4\\
  NO LINE           				&	== 	&	3, 3	&	O O  O O  &	  5\\
  CROSS              				&	== 	& 0, 0   &	X X  X X   	&  6\\\\
  \hline
\end{tabular}
\caption{Codifica input dei sensori di linea (X = tracciato nero; O = sfondo bianco)}
\label{error_normali}
\end{table}

\begin{table}
\center
\begin{tabular}{llccc}
\label{eccezioni}
								&			&	DX  SX	&	SX  	DX    &    Errore\\
 \hline\\
	GO FORWARD bis    	&	== 	&	2, 1		&	X O  O X	&  0\\
	EXCEPTION1       		&	==  	&	2, 0		&	X X  O X   &	0\\
  	EXCEPTION2        	&	== 	&	0, 1  	&	X O  X X   &  0\\
 	EXCEPTION3     	 	&	==  	&	2, 2	 	&	O X  O X	&  0\\\\
  \hline
\end{tabular}
\caption{Codifica \textsc{casi eccezionali} (X = tracciato nero; O = sfondo bianco)}
\label{eccezioni}
\end{table}

\noindent Quando il robot raggiunge un incrocio si arresta uscendo dallo stato di line follower e passa nello stato idle attendendo istruzioni dall'utente che può scegliere dalla maschera disponibile sul terminale, quindi inviare il comando tramite connessione bluetooth. Il robot torna nello stato di linefollower.

%Il modo più efficiente di progettare un controllo di un robot di questo tipo è basato sul modello di macchina a stati finiti in cui ad ogni stato corrisponde una specifica funzione del robot.
%% inserire diagramma a stati e riferimento alla figura
%Lo stato è  caratterizzato da una funzione d'ingresso (\emph{enter function}), una funzione di aggiornamento (\emph{update function}), un funziona di uscita (\emph{exit function}).
%La transizione da uno stato all'altro è soggetta al soddisfacimento di una certa condizione.


%\subsection{Stato Init}
%Stato di inizializzazione della macchina all'avvio del robot. Garantisce lo stato di fermo dei motori mediante il richiamo della funzione di stop e poi si aggiorna con la funzione start che  invoca la transizione verso lo \emph{stato Idle}.
%
%\begin{figure}[htb]
%\centering
%\subfloat[][\emph{Stato}.]{\begin{tikzpicture}[->, >=stealth', shorten >=1pt, auto, node distance=2.5cm, semithick]
%%====Definizione colori 	 paticolari===========================================
%	\definecolor{royalblue}{cmyk}{1,0.50,0,0}
%  	\definecolor{cerulean}{cmyk}{0.94,0.11,0,0}
%	\definecolor{violet}{cmyk}{0.79,0.88,0,0}
%%====Definizione stile stati================================================
%	\tikzstyle {state1}=[circle, top color=white, bottom color=orange!40, draw, violet, minimum width=1cm]
%	\tikzstyle{state2}=[circle, top color=white, bottom color=cerulean!40, draw, royalblue, minimum width=1cm]
%%====Definizione stati e posizione===========================================
%	%\draw[very thick] (2.5,0)circle[radius=4cm];
%	\draw [very thick,dashed](2.5,0) ellipse (4.5cm and 2cm);
%	\node[state1]	(A)          			{$f_{1}$};
% 	\node[state2]	(B) [right of=A]	{$f_{2}$};
% 	\node[state1]	(C) [right of=B]	{$f_{3}$};
%%====Definizione collegamenti e stile frecce================================================================
%      			 %from     stile                      scritte       	to		
%      \path 	(A)		edge	[bend left]							(B)
%      				(B)		edge	[bend left]							(C)
%      				(C)		edge	[bend left]							(B)
%	     			(B)		edge	[bend left]							(A);   
%\end{tikzpicture}}\qquad\quad
%\subfloat[][\emph{Legenda}.]{\begin{tikzpicture}[->, >=stealth', shorten >=1pt, auto, node distance=1.5cm, semithick]
%%====Definizione colori 	 paticolari===========================================
%	\definecolor{royalblue}{cmyk}{1,0.50,0,0}
%  	\definecolor{cerulean}{cmyk}{0.94,0.11,0,0}
%	\definecolor{violet}{cmyk}{0.79,0.88,0,0}
%%====Definizione stile stati================================================
%	\tikzstyle {state1}=[circle, top color=white, bottom color=orange!40, draw, violet, minimum width=1cm]
%	\tikzstyle{state2}=[circle, top color=white, bottom color=cerulean!40, draw, royalblue, minimum width=1cm]
%%====Definizione stati e posizione===========================================
%	\node[state1]	(A) [label = 0: \textsc{enter function}]           			{$f_{1}$};
% 	\node[state2]	(B) [below of=A, label = 0:\textsc{update function}]	{$f_{2}$};
% 	\node[state1]	(C) [below of=B, label = 0:\textsc{update function}]	{$f_{3}$}; 
%\end{tikzpicture}}
%\caption{Rappresentazione interna dello stato}
%\label{stato}
%\end{figure}
%
%
%\subsection{Stato Idle}
%In questo stato l'enter function richiama la funzione che stampa sul terminale i comandi disponibili per l'interazione con l'utente, poi passa nell'update function  che aspetta il comando dell'utente inviato tramite il bluetooth per la transizione nello stato successivo.

%\subsection{Stato Line Follower}
%Nel seguente stato la prima operazione che viene eseguita è una lettura mediante sensore ad ultrasuoni della presenza di eventuali ostocali sul tracciato.
%La distanza di sicurezza scelta sperimentalmente è di $15 \, cm$ valore che permette al robot di arrestarsi e ruotare su stesso evitando contatti con l'ostacolo.
%Verificata tale condizione inizia la lettura dei sensori di linea che decifrano il percorso secondo la logica presentata in tabella \ref{error_normali}. Casi eccezionali in tabella \ref{eccezioni}. Ad ogni casistica è stato associato un errore che attraverso un controllo PD appositamente 
%implementato fornisce il valore di correzione da inviare ai motori per rimanere sul tracciato.

%FUNZIONE %path_error
%  in questa funzione in ingresso vengono ricevuti
%  due interi ricavati dai metodi dei line follewer
%  e in uscita viene restituito un intero che individua la posizione del 
%  robot rispetto alla linea da seguire (error = output - setpoint).
%  Sulla base dei valori ricevuti dai sensori si ottengono i seguenti casi:\\
%\subsection{Stato Manual Command}
	\subsubsection{Controller PID}
\label{PID}

Per permettere al robot di seguire in maniera fluida il tracciato è stato implementato un controllo PID come illustrato in figura 4.

\begin{figure}[h!]
\centering
\begin{tikzpicture}[->, >=stealth', shorten >=0.1pt, auto, node distance=2cm, semithick,
								  hv path/.style={to path={-| (\tikztotarget)}},
								  tip/.style={->,shorten >=0.007pt}, 
								  vh path/.style={to path={|- (\tikztotarget)}}]

%====Definizione stile stati===============================================

	\tikzstyle{sommatore} = [circle, draw, text centered,  minimum width =0.02cm]
  	\tikzstyle{block}			=	[rectangle, draw, text centered, minimum height =1cm]
  	\tikzstyle{line} = [, draw]
%====Definizione stati e posizione==========================================
	\node[] 					(A) [label = set point]                   		{};
	\node[sommatore]  (B) [right of=A, node distance = 3cm, label = -240:$+$,label = -120:$-$]		{$\sum$};			
	\node[block]			(C) [right of=B, node distance = 3cm ]		{\textsc{pid}};
  	\node[sommatore] 	(D) [right of=C,node distance = 	3cm 	]		{} ;
  	\node[]         			(E) [right of=D,node distance = 	3cm, label = 90:uscita, label =- 90: velocita motori]  		{};
 	\node[]         			(F) [below of=B]  	{};
  	\node[]         			(G) [below of=D]  	{};
  	\node[block]         	(H) [below of=C]  	{\textsc{sensori di linea}}; 	
%====Definizione chain==================================================
    
       \path	 	(A) edge (B)
        			 	(B) edge node{errore}(C)
        				(C) edge (D);
        { [start chain]
        \chainin	(D) [join];
        { [start branch=minus]
        	\chainin (H) [join=by {vh path,tip}];
        	\chainin (B) [join=by {hv path,tip}];
        }
		\chainin(E) [join];				
    }				
\end{tikzpicture}
\caption{Schema a blocchi \textsc{PID}}
\end{figure}

\noindent Tale controllo agisce sulla velocità delle singole ruote apportando una correzione che dipende dall'errore e dai guadagni $K_p$, $K_d$, $K_i$ determinati sperimentalmente.
 L'attribuzione di un valore a questi parametri ha richiesto numerosi test, data la loro elevata influenza sull'andatura del robot. Poiché alcuni di essi risultano migliori per un determinato tracciato piuttosto che un altro si è cercato di trovare il compromesso che garantisse una maggiore flessibilità nell'affrontare differenti percorsi.
Tali guadagni permettono di calcolare la correzione da apportare ai motori mediante la seguente equazione \ref{eq}:
\begin{equation}
\label{eq}
PID_{value}=K_p\,e(t)+K_d\,\frac{de(t)}{dt}+K_i\int_0^te(\tau)d\tau
\end{equation}
Grazie a questo controllo il robot viene mantenuto allineato al percorso come si può graficamente apprezzare in un esempio di figura 5.

\begin{figure}[htb]
\centering
\label{PIDimage}
\includegraphics[width=0.65 \textwidth]{PID.jpg} 
\caption{Esempio correzione mediante \textsc{PID}}
\end{figure}

\subsubsection{Riconoscimento incroci e curve ad angolo retto}
Un incrocio viene individuato quando tutti i sensori rilevano la linea.
In tale condizione all'utente viene richiesto, mediante apposito comando, di scegliere la direzione da far intraprendere al robot. \\
Un problema particolare si è presentato nel riconoscimento delle curve ad angolo retto nelle quali, i sensori di linea, a causa della velocità del robot, uscivano brevemente dal tracciato perdendone la lettura.
Per garantire comunque una velocità sostenuta del robot è stato necessario dotare il sistema di una memoria che tenesse conto della direzione di uscita e ne garantisse il rientro dalla parte corretta. 
\subsection{Manual Control}
In tale modalità l'utente può controllare da remoto il robot tramite la connettività bluetooth.
Mediante l'utilizzo della tastiera del Pc è possibile guidare il robot nelle 4 direzioni principali agendo direttamente sui motori.
Il sistema garantisce continuamente un controllo attivo della presenza ostacoli ed in tale condizione ignora un comando errato dell'utente invertendo il moto ed evitando la collisione.
\subsubsection{Trasmissione comandi da remoto}
Per l'interfaccia con l'utente si è deciso di non utilizzare il serial monitor dell'ambiente Arduino ma di utilizzare il terminale Putty che permette un invio sequenziale dei comandi senza richiedere la pressione del tasto Enter. 
	\section{Conclusione}
\label{conclusione}
Lo sviluppo del software ha permesso di soddisfare le richieste di line following infatti il robot riesce autonomamente ad adattarsi a diversi tipi di tracciato e muoversi anche nei casi più difficili. Il setup utilizzato nel controllore \textsc{PID} permette di spingere il robot al $55\%$ della velocità massima consentita dai motori senza incorrere in slittamenti o sbandate. Permette all'utente di interagirvi con la possibilità di scegliere la direzione di svolta nel caso di incrocio e pilotarlo da remoto tramite conessione bluetooth. Infine in tutti i casi il robot evita autonomamente l'impatto con gli ostacoli.
	%\input{fran}

%===bibliografia=================================================
\nocite{*}
\bibliography{scibib}

\bibliographystyle{Science}

% Following is a new environment, {scilastnote}, that's defined in the
% preamble and that allows authors to add a reference at the end of the
% list that's not signaled in the text; such references are used in
% *Science* for acknowledgments of funding, help, etc.

% For your review copy (i.e., the file you initially send in for
% evaluation), you can use the {figure} environment and the
% \includegraphics command to stream your figures into the text, placing
% all figures at the end.  For the final, revised manuscript for
% acceptance and production, however, PostScript or other graphics
% should not be streamed into your compliled file.  Instead, set
% captions as simple paragraphs (with a \noindent tag), setting them
% off from the rest of the text with a \clearpage as shown  below, and
% submit figures as separate files according to the Art Department's
% instructions.

\clearpage
\end{document}
