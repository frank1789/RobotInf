
\begin{tikzpicture}[->, >=stealth', shorten >=1pt, auto, node distance=2.8cm, semithick]

%====Definizione stile stati================================================

	\tikzstyle{every state}=[fill, draw=none, orange ,text=white]
  	\tikzstyle{accepting}=[green!50!black, text=white]
  	\tikzstyle{initial}= [red, text=white]

%====Definizione stati e posizione===========================================
	\node[state,initial] 			(A)                    		{\textsc{A}};
 	\node[state]          			(B) [right of=A]		{\textsc{B}};
 	\node[state,accepting]	(C) [right of=B]		{\textsc{C}};
  	\node[state]         			(D) [right of=C]  		{\textsc{D}};
  	
%====Definizione collegamenti e stile frecce================================================================
      			 %from     stile                      scritte       	to		
      \path 	(A)		edge	[right]									(B)
      				(B)		edge	[bend left]							(C)
      				(C)		edge	[bend left]							(B)
      				(C)		edge[right]									(D)
      				(D)		edge[bend left]							(A);    
\end{tikzpicture}


% ====TEORIA=========================================================

\begin{tikzpicture}[->, >=stealth', shorten >=1pt, auto, node distance=2.8cm, semithick]
%====Definizione colori 	 paticolari===========================================
	\definecolor{royalblue}{cmyk}{1,0.50,0,0}
  	\definecolor{cerulean}{cmyk}{0.94,0.11,0,0}
	\definecolor{violet}{cmyk}{0.79,0.88,0,0}
%====Definizione stile stati================================================
	\tikzstyle {state1}=[circle, top color=white, bottom color=orange!40, draw, violet, minimum width=1cm]
	\tikzstyle{state2}=[circle, top color=white, bottom color=cerulean!40, draw, royalblue, minimum width=1cm]
%====Definizione stati e posizione===========================================
	\node[state1]	(A) [label = 0: \textsc{enter function}]           			{$f_{1}$};
 	\node[state2]	(B) [right of=A, label = 0:\textsc{update function}]	{$f_{2}$};
 	\node[state1]	(C) [right of=B, label = 0:\textsc{update function}]	{$f_{3}$};
%====Definizione collegamenti e stile frecce================================================================
      			 %from     stile                      scritte       	to		
      \path 	(A)		edge	[bend left]							(B)
      				(B)		edge	[bend left]							(C)
      				(C)		edge	[bend left]							(B)
	     			(B)		edge	[bend left]							(A);   
\end{tikzpicture}

%===STATE INIT========================================================

\begin{tikzpicture}[->, >=stealth', shorten >=1pt, auto, node distance=2.8cm, semithick]
%====Definizione colori 	 paticolari===========================================
	\definecolor{royalblue}{cmyk}{1,0.50,0,0}
  	\definecolor{cerulean}{cmyk}{0.94,0.11,0,0}
	\definecolor{violet}{cmyk}{0.79,0.88,0,0}
%====Definizione stile stati================================================
	\tikzstyle {state1}=[circle, top color=white, bottom color=orange!40, draw, violet, minimum width=1cm]
	\tikzstyle{state2}=[circle, top color=white, bottom color=cerulean!40, draw, royalblue, minimum width=1cm]
%====Definizione stati e posizione===========================================
	\node[state1]	(A)                    										{\textsc{stop}};
 	\node[state2]	(B) [right of=A, node distance=4cm]		{\textsc{start}};
 	\node[state1]	(C) [right of=B, node distance=4cm]		{\textsc{exit\\ function}};
%====Definizione collegamenti e stile frecce================================================================
      			 %from     stile                      scritte       	to		
      \path 	(A)		edge	[bend left]							(B)
      				(B)		edge	[bend left]							(C)
      				(C)		edge	[bend left]							(B)
	     			(B)		edge	[bend left]							(A);   
\end{tikzpicture}

%====STATE IDLE=======================================================

\begin{tikzpicture}[->, >=stealth', shorten >=1pt, auto, node distance=2.8cm, semithick]
%====Definizione colori 	 paticolari===========================================
	\definecolor{royalblue}{cmyk}{1,0.50,0,0}
  	\definecolor{cerulean}{cmyk}{0.94,0.11,0,0}
	\definecolor{violet}{cmyk}{0.79,0.88,0,0}
%====Definizione stile stati================================================
	\tikzstyle {state1}=[circle, top color=white, bottom color=orange!40, draw, violet, minimum width=1cm]
	\tikzstyle{state2}=[circle, top color=white, bottom color=cerulean!40, draw, royalblue, minimum width=1cm]
%====Definizione stati e posizione===========================================
	\node[state1]	(A)                    										{\textsc{message}};
 	\node[state2]	(B) [right of=A, node distance=4cm]		{\textsc{idle}};
 	\node[state1]	(C) [right of=B, node distance=4cm]		{\textsc{exit\\ function}};
%====Definizione collegamenti e stile frecce===============================================================
      			 %from     stile                      scritte       	to		
      \path 	(A)		edge	[bend left]							(B)
      				(B)		edge	[bend left]							(C)
      				(C)		edge	[bend left]							(B)
	     			(B)		edge	[bend left]							(A);   
\end{tikzpicture}

%===STATE LINEFOLLOWER===============================================
\begin{tikzpicture}[->, >=stealth', shorten >=1pt, auto, node distance=2.8cm, semithick]
%====Definizione colori 	 paticolari===========================================
	\definecolor{royalblue}{cmyk}{1,0.50,0,0}
  	\definecolor{cerulean}{cmyk}{0.94,0.11,0,0}
	\definecolor{violet}{cmyk}{0.79,0.88,0,0}
%====Definizione stile stati================================================
	\tikzstyle {state1}=[circle, top color=white, bottom color=orange!40, draw, violet, minimum width=1cm]
	\tikzstyle{state2}=[circle, top color=white, bottom color=cerulean!40, draw, royalblue, minimum width=1cm]
%====Definizione stati e posizione===========================================
	\node[state1]	(A)                    										{\textsc{enter\\ function}};
 	\node[state2]	(B) [right of=A, node distance=4cm]		{\textsc{line\\ follower}};
 	\node[state1]	(C) [right of=B, node distance=4cm]		{\textsc{exit\\ function}};
%====Definizione collegamenti e stile frecce================================================================
      			 %from     stile                      scritte       	to		
      \path 	(A)		edge	[bend left]							(B)
      				(B)		edge	[bend left]							(C)
      				(C)		edge	[bend left]							(B)
	     			(B)		edge	[bend left]							(A);   
\end{tikzpicture}

%===STATE STOP=======================================================

\begin{tikzpicture}[->, >=stealth', shorten >=1pt, auto, node distance=2.8cm, semithick]
%====Definizione colori 	 paticolari===========================================
	\definecolor{royalblue}{cmyk}{1,0.50,0,0}
  	\definecolor{cerulean}{cmyk}{0.94,0.11,0,0}
	\definecolor{violet}{cmyk}{0.79,0.88,0,0}
%====Definizione stile stati================================================
	\tikzstyle {state1}=[circle, top color=white, bottom color=orange!40, draw, violet, minimum width=1cm]
	\tikzstyle{state2}=[circle, top color=white, bottom color=cerulean!40, draw, royalblue, minimum width=1cm]
%====Definizione stati e posizione===========================================
	\node[state1]	(A)                    										{\textsc{stop}};
 	\node[state2]	(B) [right of=A, node distance=4cm]		{\textsc{start}};
 	\node[state1]	(C) [right of=B, node distance=4cm]		{\textsc{exit function}};
%====Definizione collegamenti e stile frecce================================================================
      			 %from     stile                      scritte       	to		
      \path 	(A)		edge	[bend left]							(B)
      				(B)		edge	[bend left]							(C)
      				(C)		edge	[bend left]							(B)
	     			(B)		edge	[bend left]							(A);   
\end{tikzpicture}

  
  
  
  
  
    
\begin{tikzpicture}[->, >=stealth', shorten >=1pt, auto, node distance=2.8cm, semithick]
 % Define block styles
	\tikzstyle{decision} = [diamond, draw, fill=blue!20, text width=4.5em, text badly centered, node distance=3cm, inner sep=0pt]
\tikzstyle{block} = [rectangle, draw, fill=blue!20, text width=5em, text centered, rounded corners,  minimum height=4em]
%	\tikzstyle{line} = [draw, -latex']
	\tikzstyle{input/output} =[trapezium, draw, minimum width=3cm,
trapezium left angle=60, trapezium right angle=120]
	\tikzstyle{start/stop} = [draw, ellipse,fill=red!20, node distance=3cm, minimum height=2em]
%%====Definizione stati e posizione===========================================
	\node	[start/stop]		(start)										{start};
	\node	[input/output]	(sensori)	[below of= start]		{prova};
    \node 	[block] (init) 					[below of= sensori]	{initialize model};
    \node	[block]	(test)					[below of= init]		{daje};
%    \node	[]()[]{};
%   \node [cloud, left of=init] (expert) {expert};
%    \node [cloud, right of=init] (system) {system};
%    \node [block, below of=init] (identify) {identify candidate models};
%    \node [block, below of=identify] (evaluate) {evaluate candidate models};
%    \node [block, left of=evaluate, node distance=3cm] (update) {update model};
%    \node [decision, below of=evaluate] (decide) {is best candidate better?};
%    \node [block, below of=decide, node distance=3cm] (stop) {stop};
%    % Draw edges
%    \path [line] (init) -- (identify);
%    \path [line] (identify) -- (evaluate);
%    \path [line] (evaluate) -- (decide);
%    \path [line] (decide) -| node [near start] {yes} (update);
%    \path [line] (update) |- (identify);
%    \path [line] (decide) -- node {no}(stop);
%    \path [line,dashed] (expert) -- (init);
%    \path [line,dashed] (system) -- (init);
%    \path [line,dashed] (system) |- (evaluate);
\end{tikzpicture}
